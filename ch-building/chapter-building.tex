\chapter{Building the Document\label{ch:building}}

This section describes how to build the document using CMake~\cite{CMake}, the
cross-platform, open-source build system and was contributed by Izaak
Beekman. This build system uses the excellent UseLATEX~\cite{UseLATEX}
CMake package to ensure that all references are resolved and defined
in the final document and that images are converted to the correct
type for the desired output. Building \LaTeX\  documents can be a
tedious and frustrating task, but this CMake based build system
attempts to remove some of the frustration. Building the
document this way is not required, an alternate build technique can
still be used.

\section{Prerequisits}
\label{sec:building:prerequisits}

This build system has \emph{not} been tested on MS Windows, but, in
theory, it should work there too. This build system requires a
working \LaTeX\  distribution, CMake~\cite{CMake}, and the appropriate
image conversion tools needed by UseLATEX~\cite{UseLATEX} to be installed.
If UseLATEX hasn't been downloaded and installed into the and have an Internet connection the
CMakeLists.txt file (which controls the build) will try to download
UseLATEX.cmake from the appropriate
source. ImageMagick~\cite{ImageMagick} is used by UseLATEX to perform
some automatic image format conversions based on the desired output
type and should be installed as well.

\section{Editing the CMakeLists.txt file}
\label{sec:building:cmakelists}

As you add and remove chapters, sections, appendices and other text
with the input and include \LaTeX\  commands, you must update the
CMakeLists.txt file to tell CMake which files are going to be used to
build the thesis. There are mechanisms to use wild-cards, also known
as globbing, to automatically collect files with the '.tex' extension,
but using this technique can cause problems with automatic dependency
analysis that is performed by CMake and UseLATEX. This list of \LaTeX\
inputs can be stored in a CMake variable and should be passed to the
``INPUTS'' argument of the ``ADD\_LATEX\_DOCUMENT'' command provided
by UseLATEX. Additionally the directories with images and figures also
need to be passed to ``ADD\_LATEX\_DOCUMENT'' as the ``IMAGE\_DIRS''
argument, to allow UseLATEX to perform the image file
conversions. Other than maintaining the list of input files and image
directories, this is the only editing of the CMakeLists.txt file that
should ever be required. A more thorough discussion of CMake and
UseLATEX is beyond the scope of this document; both tools are well
documented on their websites.

\section{Configuring and performing the build}
\label{sec:building:configuring}

UseLATEX requires an ``out of source'' build, which, in CMake
terminology, means that the document will be built in a separate
directory/folder than where you keep the input \LaTeX\  files, images,
etc. This is advantageous because it helps prevent accidentally
deleting important files when cleaning up after a \LaTeX\  build. To get
started, create a build directory then open cmake-gui and enter the
appropriate paths to the build directory and the top level source
directory that contains the CMakeLists.txt file. If working from the
command line, a short cut is to use
``cmake-gui path/to/top/level/source/directory'' or
``ccmake path/to/top/level/source/directory'' from within the build
directory. This will ensure that CMake knows where the build and
source directories are. The first time ccmake or cmake-gui is run,
``configure'' the project with the appropriate button. If prompted to
select a ``generator,'' select ``Unix Makefiles.''

Once cmake-gui or ccmake has been launched and ``configured'' for the
first time, there are some options available to help control the
build. The configuration options specific to the Princeton University
thesis template all start with ``PUTHESIS'' and currently control the
font size, whether or not to build in ``draft mode,'' whether or not a
list of figures and list of tables is included, and the output type:
``digital,'' ``proquest'' or ``print.'' The ``thesis.tex'' file has
been modified so that it can be used without CMake, but if CMake is
used to configure and build the project, \emph{no edits of the
  ``thesis.tex'' file are required to select print, digital or
  proquest modes.} If ``draft mode'' is enabled, the front matter will
be omitted, down-sampled images will be used and overfull hboxes will
be shown, again, without editing the thesis.tex file. The only other
pertinent options start with ``LATEX:'' LATEX\_SMALL\_IMAGES will
produced downsampled, 96 dpi images, without turning on ``draft mode''
and ``LATEX\_USE\_SYNCTEX,'' if turned on, will add the appropriate
synctex flag to the \LaTeX\  invocation and perform appropriate
transformations so that synctex references will point to the files in
the source directory and not their copies in the build tree. Finally,
if a required program is installed, but in a non-standard location
options exist to tell CMake/UseLATEX where those programs are
installed. Finally after the options are set to your liking run
``configure'' again to reconfigure the build for the new options, then
run the ``generate'' step, and quit out of cmake-gui.

I recommend creating a build directory for each desired output type,
rather than reconfiguring and rebuild each time you want to create a
different output. Also, as noted in the
appendix~\ref{sec:implementation:switching}, if you choose not to use
a new directory for each configuration, make sure to ``make auxclean''
after switching to a different format, before attempting to build the
document. This will delete the extra meta data files and should
allow the new format to be successfully built and the project has been
reconfigured and the build scripts regenerated by CMake.

Once the configuration and build system generation has been performed
using CMake, the document is built from the command line by running
``make.'' It will default to building a pdf document, but a list of
all possible output file formats, as well as other makefile
``targets'' can be seen by typing ``make help.'' The
final document should be created by ``make safepdf'' which uses \LaTeX\ to
produce a .dvi file and then converts it to a pdf, rather than using
pdflatex. This is not the default because it will likely take longer
to rebuild the document than using pdflatex, the default.

